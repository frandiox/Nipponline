\section{CUSL y prensa}
\label{sub:cusl_y_prensa}

A fin de mejorar la difusi'on del proyecto y motivarnos de forma adicional en su desarrollo decidimos inscribirlo
en la 8ª edici'on anual del Concurso Universitario de Software Libre 
(CUSL)\footnote{\url{http://www.concursosoftwarelibre.org/1314/}}, en el que todos los a'nos participan diferentes
proyectos de c'odigo abierto de todo el pa'is.

\begin{center}
\includegraphics[width=0.3\textwidth]{cusl}
\end{center}

Como parte de nuestra participaci'on en el concurso hemos creado y actualizado de forma regular un blog en 
Wordpress\footnote{\url{http://nipponline.wordpress.com/}}, as'i como un repositorio en 
Github\footnote{\url{https://github.com/frankdiox/Nipponline}}, plataforma que ya comentamos que usar'iamos para 
la distribuci'on del c'odigo del proyecto.

Adem'as, con intenci'on de impulsar el desarrollo de los proyectos del CUSL en Granada, la Oficina de Software Libre
organiz'o el VI Hackath'on universitario, donde tambi'en particip'o \Nipponline{}. En este evento trabajamos con
algunos colaboradores que nos ofrecieron su ayuda para avanzar en temas de documentaci'on, testeo, configuraci'on
del servidor, bocetos gr'aficos y logos para el proyecto.

A d'ia 30 de Mayo se celebr'o la fase local granadina del concurso en la que participaban los proyectos concursantes 
de la Universidad de Granada. En esta fase local el proyecto \Nipponline{} emergi'o victorioso obteniendo el primer 
puesto, lo cual fue recogido por la prensa en un art'iculo del peri'odico Ideal\cite{cuslwinner}.
