\section{Application Programming Interface (API)}
\label{sub:application_programming_interface_(api)}

Actualmente nuestra API (accesible mediante Socket.IO) se compone de 2 llamadas diferentes, necesarias para el
desarrollo de las aplicaciones actualmente funcionales. No obstante, en el futuro pretendemos ampliar el tama'no y
funcionalidad de nuestra API para atraer con mayor facilidad a colaboradores a nuestro proyecto.

Las llamadas existentes se listan a continuaci'on:

\begin{itemize}
\item \textbf{getsyllables:} No recibe ning'un argumento. Al recibir esta petici'on de un cliente, el servidor accede
a la base de datos de s'ilabas y extrae una lista en formato JSON con todas las s'ilabas disponibles, que es enviada
al cliente que hizo la petici'on.
Actualmente, esta llamada simplemente devuelve una lista de todas las s'ilabas y no recibe argumentos. No obstante,
en el futuro est'a planeado modificar esta llamada para aceptar argumentos que permiten seleccionar un subconjunto
espec'ifico de s'ilabas, en caso de que no se quieran obtener todas.
\item \textbf{getwords:} No recibe argumentos. Esta llamada es similar a la anterior, con la diferencia de que en
esta ocasi'on el servidor hace una petici'on a la base de datos para obtener un conjunto de 50 palabras distintas (si
la base de datos no dispone de suficientes palabras, como es nuestro caso actualmente, se duplican las existentes en
la lista hasta alcanzar la cantidad de 50) a devolver al cliente que hizo la petici'on.
Del mismo modo que con getsyllables, est'a planeado que esta llamada sea modificada en el futuro para aceptar
argumentos que permitan filtrar las palabras extra'idas de la base de datos, as'i como el n'umero de palabras
obtenidas.
\end{itemize}
