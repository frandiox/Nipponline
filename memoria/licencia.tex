\section{Licencia}
\label{sub:licencia}

Nipponline es un proyecto de c'odigo abierto y, como tal, debe estar protegido bajo una licencia de este tipo que 
defina las condiciones bajo las cuales se puede modificar y utilizar el software desarrollado.

Adem'as, tambi'en hemos querido licenciar todos los elementos de nuestra propia creaci'on (sobre todo, recursos gr'aficos 
para las aplicaciones) para evitar su uso sin cr'edito.

\subsection{C'odigo}
\label{sub:licencia_codigo}

La licencia escogida para nuestro proyecto Nipponline ha sido la GPLv3\footnote{\url{http://www.gnu.org/copyleft/gpl.html}}:

\begin{verbatim}

Nipponline.

Copyright (C) 2013-2014 
Francisco M. Dios <> & Sergio Balbuena <sbalbp@gmail.com>.

This program is free software; you can redistribute it and/or
modify it under the terms of the GNU General Public License as
published by the Free Software Foundation; either version 3 of the
License, or (at your option) any later version.

This program is distributed in the hope that it will be useful, but
WITHOUT ANY WARRANTY; without even the implied warranty of
MERCHANTABILITY or FITNESS FOR A PARTICULAR PURPOSE.  See the GNU
General Public License for more details.

\end{verbatim}

La licencia GPL resulta algo restrictiva, ya que fuerza a relicenciar cualquier software resultado de modificar el
original con la misma licencia GPL de este. Otras licencias m'as permisivas como MIT o BSD permiten que las 
modificaciones sobre el software original en nuevos proyectos sean licenciadas con cualquier licencia (no 
necesariamente de c'odigo abierto).

Dado que parte de nuestro proyecto depende del uso del software de terceros NodeBB (licenciado con GPL) y que es 
posible que en el futuro tengamos que realizar alguna modificaci'on sobre este para que se adapte a nuestras necesidades,
 no nos ha quedado otra opci'on que utilizar esta licencia.
 
No obstante, una ventaja del uso de la licencia GPL es que asegura que el proyecto seguir'a siendo de c'odigo abierto, 
independientemente de qui'en lo reuse o las modificaciones que realice, ya que se ver'a en la necesidad de licenciarlo 
de nuevo bajo la GPL.

\subsection{Media}
\label{sub:licencia_media}

Del mismo modo que hemos licenciado nuestro c'odigo, tambi'en hamos hecho lo propio con todos los recursos creativos 
producidos durante del desarrollo del proyecto, como son los sprites de las aplicaciones o el logo de Nipponline, 
por ejemplo.

La licencia utilizada ha sido Creative Commons\footnote{\url{http://creativecommons.org/licenses/by-nc-sa/4.0/}}, que 
enuncia lo siguiente:

\begin{verbatim}

You are free to:

Share. copy and redistribute the material in any medium or format
Adapt. remix, transform, and build upon the material

The licensor cannot revoke these freedoms as long as you
follow the license terms.

\end{verbatim}

La licencia permite modificar y redistribuir el trabajo original siempre y cuando se cumplan las siguientes 3 
condiciones:

\begin{enumerate}
\item \textbf{Atribuci'on} Debe darse cr'edito al autor original, as'i como indicar si se realizaron cambios sobre su 
obra. Adem'as, debe proporcionarse un enlace a los terminos de las licencia.
\item \textbf{No comercial} No deber'a usarse ning'un trabajo protegido con esta licencia con fines comerciales.
\item \textbf{Misma licencia} Al igual que ocurre con la licencia GPL, cualquier modificaci'on o uso del original en 
otro trabajo debe licenciarse con esta licencia.
\end{enumerate}

