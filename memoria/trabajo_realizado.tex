\section{Trabajo realizado}
\label{sub:trabajo_realizado}

En este secci'on se resume el proceso de trabajo seguido durante la realizaci'on del proyecto desde su inicio hasta
el d'ia de hoy.

\begin{itemize}
\item \textbf{Idea inicial.} Como ya hemos comentado en la introducci'on, el proyecto Nipponline ha existido desde
hace un tiempo, a lo largo del cual se ha gestado hasta el punto de convertirse en nuestro proyecto de fin de carrera.
En primer lugar hemos establecido los objetivos que queremos cumplir con este proyecto (crear una comunidad de
usuarios afines a la cultura japonesa que deseen aprender el idioma, proporcion'andoles recursos unificados y
aplicaciones) y comprobando que, efectivamente, no existe ninguna soluci'on que los cubra actualmente, dando luz
verde a la concepci'on del proyecto.
\item \textbf{An'alisis de tecnolog'ias.} El siguiente paso ha sido examinar diferentes tecnolog'ias disponibles que
se ajustaran a nuestras necesidades, permiti'endonos desarrollar el proyecto tal y como lo ten'iamos en mente. 
Obviamente, hay numerosas alternativas para cada apartado y tenemos que asegurarnos de tomar la decisi'on acertada
para no tener, en el futuro, que cambiar grandes partes del proyecto. A modo de ejemplo, una vez decidido utilizar el
software NodeBB para nuestra comunidad qued'o claro que ten'iamos que usar una base de datos no relacional, puesto
que este era el caso para NodeBB y no quer'iamos vernos forzados a instalar en el servidor numerosas bases de datos
para distintas partes del proyecto.
\item \textbf{Dise'no de la base de datos.} Antes de empezar a implementar nada, quisimos realizar un dise'no
preliminar de la base de datos del idioma, puesto que es un elemento de gran importancia para nuestro proyecto y un
mal dise'no de la misma podr'ia llevar a catastr'oficos resultados como tener que modificarla con el proyecto ya
avanzado (y junto a esta, cualquier aplicaci'on que use los datos extra'idos de ella).
De este modo, la implementaci'on de la API y de las futuras aplicaciones estar'ia condicionado por el dise'no de la
base de datos y no al contrario, como podr'ia haber ocurrido si se hubiera empezado a implementar aplicaciones sin
tener un dise'no de la base de datos inicial.
\item \textbf{Implementaci'on del servidor.} Antes de empezar a desarrollar las aplicaciones del lado del cliente se
puso un gran esfuerzo en preparar un servidor con Node.js y NodeBB, as'i como en conseguir programar sesiones de
usuario compartidas entre nuestra app de Node.js y NodeBB (cada uno se ejcuta como una app de Node.js independiente,
lo que significa que, salvo indicaci'on expl'icita, usan sesiones diferentes).
Tambi'en se desarrollan en este punto funciones para trabajar con la base de datos, que extra'igan o introduzcan
datos en esta.
Del mismo modo, tambi'en empieza en este punto la implementaci'on de las llamadas iniciales de la API, necesarias
para el desarrollo de las primeras aplicaciones.
\item \textbf{Implementaci'on de las aplicaciones.} Una vez listos el servidor, base de datos y (parcialmente) la
API, podemos empezar ya el desarrollo de las aplicaciones del cliente usando nuestra biblioteca elegida, CreateJS.
Como ya hemos mencionado, es importante que el desarrollo de las aplicaciones se realice despu'es del dise'no de la
base de datos y de la API, a fin de convertirlo en un proceso independiente del dise'no del la parte del servidor y
abrir su utilizaci'on a posibles colaboradores.
\end{itemize}
