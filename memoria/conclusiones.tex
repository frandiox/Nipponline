\section{Conclusiones}
\label{sec:conclusiones}

Con el proyecto \Nipponline{} hemos querido crear una suerte de plataforma online para aprender japon'es que aglutine 
recursos de diferente 'indole para facilitar la pr'actica del lenguaje. Tambi'en hemos optado por adherir a la funci'on 
did'actica del proyecto un elemento social, con la adici'on de una comunidad de usuarios.

Para la implementaci'on del proyecto hemos realizado un an'alisis de diferentes tecnolog'ias que pod'iamos usar a f'in de 
elegir aquellas que se adaptaran mejor a nuestras necesidades y restricciones (estamos limitados a usar software de 
c'odigo abierto, y en muchos casos la elecci'on de una herramienta puede condicionar la elecci'on de otra por motivos de 
integraci'on o licencia).

Hasta el d'ia de hoy hemos desarrollado varias aplicaciones de nivel b'asico de aprendizaje para la pr'actica de japon'es y 
hemos preparado un servidor donde se pueden probar dichas 
aplicaciones\footnote{\url{http://app.japanize.me/syllables}} \footnote{\url{http://app.japanize.me/words}}, as'i como 
la comunidad\footnote{\url{http://community.japanize.me/}}. No obstante, \Nipponline{} es un proyecto a largo plazo y 
a'un hay muchas ideas que queremos poner en pr'actica en el futuro en forma de nuevas aplicaciones.

Finalmente, tambi'en hemos preparado una API, actualmente accesible con Javascript v'ia Socket.IO, que permite a 
desarrolladores externos al proyecto colaborar en el desarrollo de nuevas a aplicaciones y/o desarrollar las suyas 
propias.
